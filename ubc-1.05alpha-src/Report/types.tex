\section{Types}
\label{types}
\label{abstracttypes}
A type is defined as a collection of operation signatures, where
each operation signature includes the operation name, and the names and
types of its arguments and results. Types, being objects, are
first-class citizens in Emerald. Each type object exports a function without
arguments called {\em getSignature} that returns an object of the predefined
\kw{Signature} type. In other words, any object that conforms to 
the following type:
{\small\it\begin{tabbing}
xxx\=xxx\=xxx\=xxx\=xxx\=xxx\=xxx\=xxx\=xxx\=xxx\=xxx\=xxx\=xxx\=\+\kill%
\kw{immutable} \kw{typeobject} \tn{type}\+\\*{}%
  \kw{function} getSignature \returns{} \/\LB{}\tn{Signature}\/\RB{}\-\\*{}%
\kw{end} \tn{type}\\*{}%
\end{tabbing}\vspace{-2\topsep}}

is a type.  Note that each object with type \tn{signature} has a getSignature
operation that returns self, thus Signatures are Types.

\subsection{Type Constructors}
\label{type constructors}
Signatures are created
using type constructors.  A type constructor has the following
structure:
\begin{quote}\it\begin{tabular}{lcl}
typeConstructor &$::=$& \opt{\kw{immutable}} \kw{typeobject} typeIdentifier \\
& &\hspace{0.5in}\oseq{operationSignature} \\
& &\kw{end} typeIdentifier
\end{tabular}\end{quote}
Operation signatures have been defined in
Section~\ref{operationSignature}, however in type constructors, the
identifiers in parameter declarations may be omitted.
An immutable type implies that its objects are abstractly immutable, i.e.
its objects cannot change their state over time. For example, the predefined
type  
\kw{Integer} is immutable because its objects represent integer values which
cannot change; for instance, the integer 3 cannot be changed to the integer
4.

\subsection{Syntactictypeof and typeof}
\label{syntactictypeof}
\label{syntactictype}
The syntactictypeof operator evaluates to the statically determined type of
an expression.  That is, the type of the expression as it can be determined
by the compiler.  This operation is always evaluated at compile time.  In
contrast the \kw{typeof} operator returns the most
accurate type of the expression at run time, which can be a wider type than
the syntactic type, if either implicit or explicit narrowing has occurred.
This most accurate type is 
also called the {\it best fitting} type of the expression.
See section \ref{typeofobjectconstructor}.  It is always the case that for
any expression {\it e}, \kw{typeof} {\it e} \conforms \kw{syntactictypeof} {\it e}.

\subsection{Conformity}
\label{conformity}
Conformity is the basic relationship between types.  
A type 
{\it S}
conforms to a type 
{\it T}
(written 
{\it S}
\conforms{} 
{\it T}\/) 
if:
\begin{enumerate}
  \item{} {\it S} is immutable if {\it T} is immutable.
  \item{} For each operation $o_T$ in {\it T},
    there exists an operation $o_S$ in {\it S} with the same name and number
    of arguments, and
  \item{} $o_T$ and $o_S$ have the same number of results, and
  \item{} The  types of the results of $o_S$ operations conform to the
    types of the results of $o_T$, and 
  \item{} The  types of the arguments of $o_T$ conform to the  types of the
    arguments of $o_S$ (i.e., arguments must conform in the opposite
    direction). 
\end{enumerate}

If either {\it S} or {\it T} is recursive (the definition of at least one of
its operations uses its own name), then the previous checks must be
performed under an assumption that {\it S} \conforms{} {\it T}.

This simple description of conformity suffices for all invocations that
do not involve parametric polymorphism.  Discussion of polymorphic
operations and the extensions to the type checking rules required to type
check them is deferred until Section \ref{polymorphism}.

Some types in the system are exceptions to
the standard rules for conformity.  For ensuring
correctness, types such as
\tn{Boolean}, \tn{Condition}, \tn{Node}, \tn{Signature} and \tn{Time} must
be implemented only by the system. For performance enhancement,
the types
\tn{Character}, \tn{Integer}, \tn{Real}, and \tn{String} are also
restricted to be implemented only by the system.

\subsection{Polymorphism}
\label{polymorphism}
Inclusion polymorphism, where one type is a subtype of another and can be
used in any context in which the supertype is expected, is fundamental to
Emerald.  Its type system is defined so that the subtyping relation
(conformity) is as large as possible while still guaranteeing safety: an
operation will never be performed on an object that doesn't implement that
operation.

Emerald also supports parametric polymorphism, where types are either
explicitly or implicitly passed as arguments to invocations.  The cardinal
rule regarding types is that it must be possible for the compiler at compile
time to determine the value of every expression that is used in a position
where a type is required.  Such positions include variable, constant, and
parameter declarations, and the second argument to view and restrict
expressions.

Since types are objects, no special type parameterization form is necessary
in Emerald; types are passed to operations in the same way as are other
objects.   The operation signature definition syntax is supplemented with
three clauses that permit:
\begin{itemize}
  \item{}  the introduction of dependent types whose values
    depend on the value of type arguments (the where clause)
  \item{} the declaration of type variables that may take on the value of
    the types of other arguments (the for all clause)
  \item{} the specification of constraints on the values of type
    variables (the such that clause).  
\end{itemize}

\subsubsection{For all clause}
\begin{quote}\it\begin{tabular}{lcl}
forallClause & $::=$& \kw{forall} identifier \\
\end{tabular}\end{quote}
The for all clause defines the {\it identifier} as a type variable without
constraint.  If the identifier is not defined elsewhere in the operation
signature, the for all clause also serves to define it.  The type variable
will take as its value the type value positionally assigned to it during
invocation.  The primary purpose of the for all clause is to capture the
type of some other argument to an invocation, as in the following example of
the polymorphic identity function which returns its argument and in which
the type of the result is the same as the type of the argument:

{\small\it\begin{minipage}{\textwidth}\begin{tabbing}
xxx\=xxx\=xxx\=xxx\=xxx\=xxx\=xxx\=xxx\=xxx\=xxx\=xxx\=xxx\=xxx\=\+\kill%
  \kw{function} identity\/\LB{}a \CO{} t\/\RB{} \returns{} \/\LB{}b \CO{} t\/\RB{}\+\\*{}%
    \kw{forall} t\\*{}%
    b \assign{} a\-\\*{}%
  \kw{end} identity
\end{tabbing}\end{minipage}}

The for all clause can also be used to introduce a type variable whose value
is further constrained by a such that clause.

\subsubsection{Where clause}
\begin{quote}\it\begin{tabular}{lcl}
whereClause & $::=$& \kw{where} identifier {\assign} typeExpression
\end{tabular}\end{quote}
A where clause is semantically equivalent to a constant declaration.  It
defines the {\it identifier} to have the value of the given {\it
typeExpression}.  The type expression is evaluated during type checking of
invocations after type values have been bound to the argument identifiers
defined in the operation signature and any type variables have also been
bound to their values.  This permits the construction of dependent types
whose values depend on the values of type arguments to the invocation, as
demonstrated in the following example:

{\small\it\begin{minipage}{\textwidth}\begin{tabbing}
xxx\=xxx\=xxx\=xxx\=xxx\=xxx\=xxx\=xxx\=xxx\=xxx\=xxx\=xxx\=xxx\=\+\kill%
  \kw{function} makeVector\/\LB{}arg \CO{} argType\/\RB{} \returns{} \/\LB{}res \CO{} resType\/\RB{}\+\\*{}%
    \kw{forall} argType\\*{}%
    \kw{where} resType \assign{} \tn{Vector}.of\/\LB{}argType\/\RB{}\\[1.0ex]{}%
    res \assign{} resType.create\/\LB{}10\/\RB{}\\*{}%
    \kw{for} i \CO{} \tn{Integer} \assign{} 0 \kw{while} i $<$ 10 \kw{by} i \assign{} i $+$ 1\+\\*{}%
      res\/\LB{}i\/\RB{} \assign{} arg\-\\*{}%
    \kw{end} \kw{for}\-\\*{}%
  \kw{end} makeVector
\end{tabbing}\end{minipage}}


\subsubsection{Such that clause}
\begin{quote}\it \begin{tabular}{lcl}
suchthatClause & $::=$ & \kw{suchthat} identifier \matches{} typeLiteral
\end{tabular}\end{quote}

The such that clause allows the possible values that may be taken on by a
type variable to be constrained.  Any value bound to the given {\it
identifier} (which must be a type variable) must match the {\it
typeExpression}.  This clause allows the programmer to require that an
argument type have a particular collection of operations, as demonstrated in
the following example:

{\small\it\begin{minipage}{\textwidth}\begin{tabbing}
xxx\=xxx\=xxx\=xxx\=xxx\=xxx\=xxx\=xxx\=xxx\=xxx\=xxx\=xxx\=xxx\=\+\kill%
  \kw{function} inOrder\/\LB{}a \CO{} t, b \CO{} t, c \CO{} t\/\RB{} \returns{} \/\LB{}r \CO{} \tn{Boolean}\/\RB{}\+\\*{}%
    \kw{forall} t\\*{}%
    \kw{\kw{such}\kw{that}} t \conforms{} \kw{typeobject} comparable\+\\*{}%
      \kw{function} $<=$\/\LB{}comparable, comparable\/\RB{} \returns{} \/\LB{}\tn{Boolean}\/\RB{}\-\\*{}%
    \kw{end} comparable\\[1.0ex]{}%
    r \assign{} a $<=$ b \kw{and} b $<=$ c\-\\*{}%
  \kw{end} inOrder
\end{tabbing}\end{minipage}}


\subsection{Typechecking operation definitions}
In the presence of parametric polymorphism, some of the identifiers defined
in the signature of an operation definition will be type variables, which
will take on 
different values for each invocation of the operation.  In type checking the
body of the operation, we can assume only that the value of each type
variable will match its constraint.  Therefore we type check the body of the
operation with all type variables bound to their constraints.  Once the
operation body has been shown to be type correct under this assumption, it
will be type correct for every invocation since the actual type bound to each
type variable must match the constraint on that type variable.

\subsection{Typechecking invocations}
In the presence of parametric polymorphism, the simple rules for
typechecking invocations (that the type of each argument expression must
conform to the type of its corresponding formal parameter) is insufficient.
Therefore, when typechecking an invocation involving type variables, the
following steps are performed:
\begin{enumerate}
  \item{} Bind each type variable to its corresponding type value from the
    actual arguments.  These may be the arguments themselves, or in the case
    of type variables introduced by for all clauses, the types of the
    arguments. 
  \item{} Create the dependent type introduced in each where clauses in the
    operation signature.  All of these objects are created simultaneously so
    that recursive type structures can be successfully created.
  \item{} Check that the value bound to each type variable matches the
    constraint on that type variable.
  \item{} Check that the type of each argument conforms to the type of its
    corresponding formal parameter.
  \item{} Determine the result types of the invocation by using the current
    values of any type variables or dependent type identifiers.
\end{enumerate}
\subsection{Conformity revisited}
Parametric polymorphism complicates the conformity rules as well, since it
introduces type variables and constraints on them.  The previous rules for
conformity still must be satisfied, but in addition:
\begin{enumerate}
  \item[0]{} If either or both of {\it S} and {\it T} is a bound type
    variable, then consider the values bound to them rather than the type
    variable when checking the other rules.
  \item[5]{} If {\it T} is an unbound type variable then {\it S} must be
    also, and the constraint on {\it S} must match the constraint on {\it T}.
\end{enumerate}
\subsection{Matches}
The matches relation (\matches{}) between types is very similar to the
conformity relation (\conforms{}).  In fact, if the types being considered
are not recursive then the two relations are equivalent.  When checking the
conformity of two recursive types ({\it S} and {\it T}) we must
first assume that {\it S} \conforms{} {\it T}.  When checking whether {\it
S} matches {\it T} we assume instead that {\it S} and {\it T} are equivalent
types ---  that {\it S} \conforms{} {\it T} and that {\it T} \conforms{} {\it
S} --- because if the types do match then they will be bound together (since
matches only comes into play in the presence of type variables).

\subsection{Polymorphism Example}
\label{polymorphismexample}
\begin{figure}[tbp]
\begin{center}
{\small\it\begin{minipage}{\textwidth}\begin{tabbing}
xxx\=xxx\=xxx\=xxx\=xxx\=xxx\=xxx\=xxx\=xxx\=xxx\=xxx\=xxx\=xxx\=\+\kill%
\kw{const} Set \assign{} \kw{immutable} \kw{object} Set\+\\*{}%
  \kw{export} \kw{function} of\/\LB{}eType \CO{} \tn{type}\/\RB{} \returns{} \/\LB{}result \CO{} NewSetType\/\RB{}\+\\*{}%
    \kw{\kw{such}\kw{that}}\+\\*{}%
      eType \matches\+\\*{}%
	\kw{immutable} \kw{typeobject} eType\+\\*{}%
	  \kw{function} $=$\/\LB{}eType\/\RB{} \returns{} \/\LB{}\tn{Boolean}\/\RB{}\-\\*{}%
	\kw{end} eType\-\-\\*{}%
    \kw{where}\+\\*{}%
      NewSetType \assign{}\+\\*{}%
	\kw{immutable} \kw{typeobject} NewSetType\+\\*{}%
	  \kw{operation} empty \returns{} \/\LB{}NewSet\/\RB{}\\*{}%
	  \kw{operation} singleton\/\LB{}eType\/\RB{} \returns{} \/\LB{}NewSet\/\RB{}\\*{}%
	  \kw{operation} create\/\LB{}sequenceOfeType\/\RB{} \returns{} \/\LB{}NewSet\/\RB{}\-\\*{}%
	\kw{end} NewSetType\-\-\\*{}%
    \kw{where}\+\\*{}%
      sequenceOfeType \assign{}\+\\*{}%
	\kw{immutable} \kw{typeobject} sequenceOfeType\+\\*{}%
	  \kw{function} lowerbound \returns{} \/\LB{}\tn{Integer}\/\RB{}\\*{}%
	  \kw{function} upperbound \returns{} \/\LB{}\tn{Integer}\/\RB{}\\*{}%
	  \kw{function} getElement\/\LB{}\tn{Integer}\/\RB{} \returns{} \/\LB{}eType\/\RB{}\-\\*{}%
	\kw{end} sequenceOfeType\-\-\\*{}%
    \kw{where}\+\\*{}%
      NewSet \assign{}\+\\*{}%
	\kw{immutable} \kw{typeobject} NewSet\+\\*{}%
	  \kw{function} contains\/\LB{}eType\/\RB{} \returns{} \/\LB{}\tn{Boolean}\/\RB{}\\*{}%
	  \kw{function} $+$\/\LB{}NewSet\/\RB{} \returns{} \/\LB{}NewSet\/\RB{}\\*{}%
	  \kw{function} $*$\/\LB{}NewSet\/\RB{} \returns{} \/\LB{}NewSet\/\RB{}\\*{}%
	  \kw{function} $-$\/\LB{}NewSet\/\RB{} \returns{} \/\LB{}NewSet\/\RB{}\\*{}%
	  \kw{function} cardinality \returns{} \/\LB{}\tn{Integer}\/\RB{}\-\\*{}%
	\kw{end} NewSet\-\-\\*{}%
    result \assign{} \+\\*{}%
      \kw{object} SetCreator\+\\*{}%
	\kw{export} \kw{operation} create\/\LB{}v \CO{} sequenceOfeType\/\RB{}\returns{} \/\LB{}result \CO{} NewSet\/\RB{}\+\\*{}%
	  result \assign{}\+\\*{}%
	    \kw{object} NewSet\+\\*{}%
	      \kw{const} repType \assign{} \tn{Vector}.of\/\LB{}eType\/\RB{}\\*{}%
	      \kw{var} rep \CO{} repType\\[1.0ex]{}%
	      \cd{} The implementation of the operations and functions.\-\-\\*{}%
	  \kw{end} NewSet\-\\*{}%
	\kw{end} create\\*{}%
	\kw{export} \kw{operation} empty \returns{} \/\LB{}r \CO{} NewSet\/\RB{}\+\\*{}%
	  r \assign{} self.create\/\LB{}\kw{nil}\/\RB{}\-\\*{}%
	\kw{end} new\\*{}%
	\kw{export} \kw{operation} singleton\/\LB{}e \CO{} eType\/\RB{} \returns{} \/\LB{}r \CO{} NewSet\/\RB{}\+\\*{}%
	  r \assign{} self.create\/\LB{}{e}\/\RB{}\-\\*{}%
	\kw{end} singleton\-\\*{}%
      \kw{end} SetCreator\-\-\\*{}%
  \kw{end} of\-\\*{}%
\kw{end} Set
\end{tabbing}\end{minipage}}
\end{center}
\caption{A Polymorphic Set Object}
\label{polymorphicset}
\end{figure}
To demonstrate the polymorphism present in \emd{}, a polymorphic {\em Set}
object is presented in Figure~\ref{polymorphicset}. {\em Set} 
has an
operation {\em of} that takes a type as an argument and returns
an object that can be used as the abstract type of, as well as a creator
of, sets of things conforming to the original argument to the operation
{\em of}.  The element type for a set (the type passed to the {\it of}
function) must be immutable and must
implement an $=$ operation that returns a \kw{Boolean} object. With this
{\em Set} definition, we can define creators for sets of integers and
strings as:
{\small\it\begin{tabbing}
xxx\=xxx\=xxx\=xxx\=xxx\=xxx\=xxx\=xxx\=xxx\=xxx\=xxx\=xxx\=xxx\=\+\kill%
\kw{const} IntSet \defined{} Set.of\/\LB{}\tn{Integer}\/\RB{}\\*{}%
\kw{const} StringSet \defined{} Set.of\/\LB{}\tn{String}\/\RB{}\\*{}%
\end{tabbing}\vspace{-2\topsep}}
\noindent
and we can create singleton sets of integers and strings as:
{\small\it\begin{tabbing}
xxx\=xxx\=xxx\=xxx\=xxx\=xxx\=xxx\=xxx\=xxx\=xxx\=xxx\=xxx\=xxx\=\+\kill%
\kw{const} i \defined{} IntSet.singleton\/\LB{}6\/\RB{}\\*{}%
\kw{const} s \defined{} StringSet.singleton\/\LB{}\terminal{"}abc\terminal{"}\/\RB{}\\*{}%
\end{tabbing}\vspace{-2\topsep}}
