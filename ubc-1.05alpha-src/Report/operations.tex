\section{Operations}
\label{operations}
\emd{} objects communicate with one another only through the invocation of
operations. This section describes the definition and invocation of
operations.

\subsection{Defining operations}
\label{operationSignature}
\begin{quote}\it\begin{tabular}{lcl}
operationSignature &$::=$& operationKind operationName \opt{parameterList} \\
& & \hspace{0.5in}\opt{\lrq{\returns} resultList} \oseq{clause} \\
operation &$::=$& \opt{\kw{export}} operationSignature\\
& & \hspace{0.5in}blockBody \\
& & \kw{end} operationName \\
operationKind &$::=$& \kw{op} $|$ \kw{operation} $|$ \kw{function} \\
parameterList &$::=$& \terminal{[} \sseq{parameter}{\terminal{,}} \terminal{]} \\
resultList &$::=$& parameterList\\
parameter &$::=$&	\opt{\kw{attached}} \opt{identifier :} type \\
clause &$::=$ & whereClause \\
	& $|$ & forallClause \\
	& $|$ & suchthatClause
\end{tabular}
\end{quote}
\emd{} provides two kinds of operations: procedural and functional. 
Procedural operations are heralded by the keyword \kw{operation} or \kw{op},
while the keyword \kw{function} indicates a
functional operation.  In declaring a functional operation, 
the programmer asserts that the operation
is side-effect free, i.e., the abstract state of the system
is not modified by the execution of the operation\footnote{Note that \emd{}
does not rule out the possibility of the operation having concrete
side-effects (sometimes termed {\em beneficial} or {\em benevolent}
side-effects).}.  Note that the burden is on the programmer;  the Emerald
system may perform optimizations on function invocations that are incorrect if
the operation has side effects.

The operation signature (cf.~Section~\ref{abstracttypes})
includes the operation name and the number, names
and abstract types of its arguments and results.  
An object may implement multiple operations with the same name, provided
that the number of arguments that they accept is different.
Where clauses serve to
introduce new names for types whose scope is the entire operation signature
(and body if present).  For all clauses declare new type identifiers.
Such that clauses impose
constraints on the formal parameters.
These clauses are primarily useful for the implementation of
polymorphic types and are defined in Section~\ref{polymorphism}.

\subsection{Parameter Passing}
The \emd{} language uses call-by-object-reference semantics for all
invocations, local or remote.  That is, a reference to the argument object
is passed to the called procedure.  There are no restrictions on the types
of objects that can be passed.

\subsection{Making Invocations}
\begin{quote}\it\begin{tabular}{lcl}
procedureInvocation &$::=$& expression \terminal{.} operationName \opt{argumentList} \\
operationName &$::=$& identifier  $|$  operator \\
argumentList  &$::=$& \terminal{[} \sseq{argument}{\terminal{,}} \terminal{]} \\
argument & $::=$ &  \opt{\kw{move} $|$ \kw{visit}} expression
\end{tabular}\end{quote}

An invocation specifies the target object, the operation to be
invoked, and any arguments. When
an invocation returns results, they are assigned to variables using an
assignment (see~Section~\ref{assignment statement}).

\label{call by move}
Because Emerald objects are mobile, it may be possible to optimize an
invocation by
avoiding many
remote references by moving argument objects to the site of a
remote invocation. 
The keywords \kw{move} and \kw{visit} suggest that the
expression be physically moved to the same node as the invoked object;
\kw{visit} further suggests that the expression be moved back when the
invocation returns.
These two 
parameter passing modes are called
{\em call-by-move} and {\em call-by-visit} respectively.
Neither mode
affects the location-independent semantics of
the invoked operation.


Executing an operation invocation involves:
\begin{itemize}
\item{} evaluating the invocation target expression,
\item{} evaluating the argument objects and then positionally
assigning them to the formal parameters of the operation,
\item{} executing the body of the operation in the
context of the target object of the invocation, and
\item{} returning the final values of any output parameters of the
invocation.
\end{itemize}
