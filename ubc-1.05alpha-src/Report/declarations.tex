\section{Declarations}
\label{declarations}
Every identifier used in \emd{} must be
declared.  There are two (general purpose) declarative forms:
one for constants and one for variables.

\begin{quote}\it\begin{tabular}{lcl}
declaration &$::=$& variableDeclaration \\
& $|$ & constantDeclaration \\
constantDeclaration &$::=$& \opt{\kw{attached}} \kw{const} identifier
    \opt{: type} initializer \\
variableDeclaration &$::=$& \opt{\kw{attached}} \kw{var} 
    identifierList : type \opt{initializer} \\
identifierList &$::=$& \sseq{identifier}{\terminal{,}} \\
initializer &$::=$& \terminal{\assign} expression
\end{tabular}\end{quote}

A constant declaration introduces an identifier that refers to a
single object throughout its lifetime.  If the optional type clause is
present, the type of the identifier is the value of the {\it type}
expression, and the type of the {\it initializer} expression must
conform 
to the type of the identifier.  If the type clause is not present, the type
of the identifier is the syntactic type of the {\it initializer}
(cf.~Section~\ref{syntactictype}).
While a
constant identifier always refers to the same object;  the object's
state may change if it is mutable.

A variable declaration introduces a new variable for each element of the
{\it identifierList}.  The type of each identifier is the value of the {\it
type} expression.
When an
initializer clause is present the type of the {\it initializer} must conform
to the type of the identifiers, and each variable in the list is assigned the
value of the {\it initializer} expression.  When not explicitly initialized,
variables initially name the object \kw{nil}.
A variable identifier may have its value changed by assignment 
(cf.~Section~\ref{assignment statement}). 

The optional \kw{attached} permits the programmer to provide one-way
attachment between objects; the relocation of an object
additionally relocates all co-located objects attached to it
(cf.~Section~\ref{move statement}). 

A declaration is an executable construct, and the initialization of
constants and variables is performed each time the declaration is executed.

\subsection{Scope}
An identifier name is visible
throughout the scope in which it is declared, not just textually after
that declaration.  The following constructs open new scopes for
identifiers, and identifiers are imported implicitly into nested
scopes where they are not redefined:
\begin{itemize}
  \item{} if, elseif, and else clauses (cf. Section \ref{selection stmt})
  \item{} loop statement bodies (cf. Section \ref{loop stmt})
  \item{} blocks, unavailable, and failure handlers (cf. 
            Section \ref{compound statement})
  \item{} operation definitions and signatures (cf. Section 
            \ref{operationSignature})
  \item{} typeobject constructors (cf. Section \ref{type constructors})
  \item{} object constructor, process, initially-block and
           recovery-block definitions (cf. Section \ref{object constructor})
\end{itemize}

Since object constructors and typeobject constructors create new objects
that are independent of their enclosing referencing environment (closures), 
identifiers imported into these constructs
are treated specially.  When the type or object constructor is
executed, all imported identifiers are made constant.
Throughout the lifetime of the created type or object, these
identifiers will have the values that they had when the object constructor
was executed. Consider the following example:

{\small\it\begin{minipage}{\textwidth}\begin{tabbing}
xxx\=xxx\=xxx\=xxx\=xxx\=xxx\=xxx\=xxx\=xxx\=xxx\=xxx\=xxx\=xxx\=\+\kill%
\kw{for} i \CO{} \tn{Integer} \assign{} 0 \kw{while} i $<$ 10 \kw{by} i \assign i $+$ 1\+\\*{}%
  o \assign{} \kw{object} trivial\+\\*{}%
    \kw{export} \kw{operation} getI \returns{} \/\LB{}r \CO{} \tn{Integer}\/\RB{}\+\\*{}%
      r \assign{} i\-\\*{}%
    \kw{end} getI\-\\*{}%
  \kw{end} trivial\-\\*{}%
\kw{end} \kw{for}
\end{tabbing}\end{minipage}}

This loop creates ten identical objects, except that the value
of the identifier {\it i} is different for each object.  Once the
first object (whose {\it i} $=$ 0) is created, changes to the loop
control variable {\it i} are not visible to it, as the {\it i} that it
sees was made constant when that object was created.

While identifiers are in scope for the entire block in which they are
declared (even textually before their declaration), they are normally given
values as their declarations are executed.  For example, the following
operation definition is correct as far as scope is concerned, but the
final value of {\it x} will be \kw{nil} while the final value of of {\it y}
will be 6.

{\small\it\begin{minipage}{\textwidth}\begin{tabbing}
xxx\=xxx\=xxx\=xxx\=xxx\=xxx\=xxx\=xxx\=xxx\=xxx\=xxx\=xxx\=xxx\=\+\kill%
\kw{operation} nonintuitive\+\\*{}%
  \kw{const} x \assign{} y\\*{}%
  \kw{const} y \assign{} 6\-\\*{}%
\kw{end} nonintuitive
\end{tabbing}\end{minipage}}

There are two exceptions to this rule:
\begin{enumerate}
  \item{} In a compilation unit, all exported identifiers get their values
    simultaneously.  This permits the construction of mutually recursive
    object structures.
  \item{} When typechecking an invocation of a polymorphic operation, the
  values of all identifiers introduced in whee and for all clauses get their
  values simultaneously.  This permits the construction of the mutually
  recursive type objects that are often necessary when creating polymorphic
  operations.
\end{enumerate}
