\section{Objects}
\label{objects}
Emerald provides a single general purpose object constructor which creates
all objects, as well as a number of syntactic shorthands for commonly
occurring usage patterns.

\subsection{Object Constructors}
\label{object constructor}
An object constructor defines the complete representation, operations, and
active behaviour of
a single object. Objects are created when an
object constructor is executed. In other words, object constructors
are expressions. The form of a constructor shown below
demonstrates its generality, i.e.,
all \emd{} objects may be defined using this feature.
\begin{quote}\it\begin{tabular}{lcl}
objectConstructor &$::=$& \opt{\kw{immutable}} \opt{\kw{monitor}} \kw{object} identifier \\
& &\hspace{0.5in}\oseq{declaration} \\
& &\hspace{0.5in}\oseq{operation $|$ initially $|$ process $|$ recovery} \\
& &\kw{end} identifier \\[1ex]

process &$::=$& \kw{process} \\
& &\hspace{0.5in}blockBody \\
& & \kw{end} \kw{process} \\[1ex]

initially &$::=$&\kw{initially} \\
& &\hspace{0.5in}blockBody \\
& &\kw{end} \kw{initially} \\[1ex]

recovery &$::=$&\kw{recovery} \\
& &\hspace{0.5in}blockBody \\
& &\kw{end}\kw{recovery}
\end{tabular}\end{quote}

Each object in \emd{} owes its existence to either an implicit or explicit
execution of an object constructor. The object constructor provides the
necessary information about the object's implementation, i.e.,
\begin{itemize}
\item{} Representation declarations for data and processes that are
contained in the object. 
\item{} A collection of operation bodies containing both the signature as
well as the implementation of each operation that the object is capable of
executing.
\end{itemize}
The type of an object constructor expression is determined by including in
the type the signature of every exported operation in the constructor.  This
defines the {\it 
best fitting type}, and may be retrieved during execution by the \kw{typeof}
expression.\label{typeofobjectconstructor}

\subsubsection{Initialization}
When an object constructor is executed, the newly created object is
initialized by performing the following steps in order:
\begin{enumerate}
  \item{} Initialize any implicitly created constants holding the values
    of imported identifiers.
  \item{} Initialize all variables and constants declared in the constructor
    in textual order.
  \item{} Execute the declarations and statements in the initially section,
    if present.
\end{enumerate}
Any attempt to invoke any operation on the object is deferred until
its initialization is complete.  The only exception to his rule is that
invocations of the object by itself during initialization are allowed.

Once initialization is complete, the process defined by the object
constructor is started and then the execution of the object constructor
expression terminates.  The value of the expression is a reference to the
newly created object.  

\subsubsection{Recovery}
When an Emerald node on which objects have previously checkpointed recovers,
all checkpointed object have their state restored as of the time of the most
recently completed checkpoint, and then the declarations and statements in
the recovery section of the object constructor that caused the creation of
the object are executed.  During recovery, invocations on the object are
deferred as during initialization.  When the execution of the recovery block
is complete, the process defined by the object constructor is started anew.

\subsubsection{Object creators}
An object whose primary purpose is the creation of other object is termed an
object creator.  No additional language mechanisms are needed to program
object creators; one simply nests one object constructor inside another, but
see also Section~\ref{classes}.

\subsection{Objects as Types}
\label{objectsastypes}
Type constructors are the basic method for
constructing abstract types in \emd{} (cf.~Sec\-tion~\ref{abstracttypes}). 
Since typing in Emerald is based entirely on the signatures of operations,
any object which conforms to the type

{\it\begin{minipage}{\textwidth}\begin{tabbing}
xxx\=xxx\=xxx\=xxx\=xxx\=xxx\=xxx\=xxx\=xxx\=xxx\=xxx\=xxx\=xxx\=\+\kill%
\kw{immutable} \kw{typeobject} \tn{type}\+\\*{}%
  \kw{function} getSignature \returns{} \/\LB{}\tn{Signature}\/\RB{}\-\\*{}%
\kw{end} \tn{type}
\end{tabbing}\end{minipage}}

is a type.  Thus objects which serve other useful purposes can also be used
as types.  Object creators in particular can take advantage of this to
allow a single object to serve as both a creator and a type.

\subsection{Classes}
\label{classes}
Emerald does not have a notion of {\it class\/}.  That is, it is not
possible to distinguish a class object from some other object.  Or, stated
differently again, Emerald does not have a type {\it class} which class
objects conform to but other objects do not.  However, Emerald does have a
syntactic construct called a class that provides the functionality normally
expected of classes.
\begin{quote}\it\begin{tabular}{lcl}
class &$::=$& \opt{\kw{immutable}} \opt{\kw{monitor}} \kw{class} identifier \\
& & \hspace{0.5in} \opt{\terminal{(} baseClass \terminal{)}} \opt{parameterList}\\
& & \hspace{0.5in} \oseq{classoperation} \\
& & \hspace{0.5in} \oseq{declaration} \\
& & \hspace{0.5in} \oseq{operation $|$ initially $|$ process $|$ recovery} \\
& & \kw{end} identifier\\
baseClass &$::=$& identifier\\
classoperation  &$::=$& \kw{class} operation
\end{tabular}\end{quote}

Classes are expanded syntactically into two nested object constructors.  The
outer object (the class, factory, or creator object) is immutable and
declares a single constant, a \tn{Signature} which represents the type of
the instances and 
whose name is derived from the name of the type with the string ``type''
appended.  This signature object contains the
signature of each operation that is exported from the inner object
constructor (which defines the instances), and is the {\it best fitting}
type of those instances.
The class exports operations {\it getSignature} and {\it create} in 
addition to the class operations defined by the programmer.  
The
{\it getSignature} operation returns the signature constant described above.
The {\it
parameterList} specifies the parameter list to the {\it create} operation;
the body of the create operation is a single assignment statement which
returns the result of executing the inner object constructor.
The inner object constructor is given the name of the class prefixed with
the string ``a'' (or ``an'' as appropriate).  
The rest of the components of
the class construct become the body of the inner object constructor and
thereby define the class's instances.

This is most easily understood through examples.  Suppose we write the
following declaration: 

{\small\it\begin{minipage}{\textwidth}\begin{tabbing}
xxx\=xxx\=xxx\=xxx\=xxx\=xxx\=xxx\=xxx\=xxx\=xxx\=xxx\=xxx\=xxx\=\+\kill%
\kw{const} Complex \assign{} \kw{immutable} \kw{class} Complex\/\LB{}r \CO{} \tn{Real}, i \CO{} \tn{Real}\/\RB{}\+\\*{}%
  \kw{class} \kw{export} \kw{operation} fromReal\/\LB{}a \CO{} \tn{Real}\/\RB{} \returns{} \/\LB{}e \CO{} Complex\/\RB{}\+\\*{}%
    e \assign{} self.create\/\LB{}a, 0.0\/\RB{}\-\\*{}%
  \kw{end} fromReal\\*{}%
  \kw{export} \kw{function} $+$\/\LB{}other \CO{} Complex\/\RB{} \returns{} \/\LB{}e \CO{} Complex\/\RB{}\+\\*{}%
    e \assign{} Complex.create\/\LB{}other.getReal $+$ r, other.getImag $+$ i\/\RB{}\-\\*{}%
  \kw{end} $+$\\*{}%
  \kw{export} \kw{function} getReal \returns{} \/\LB{}e \CO{} \tn{Real}\/\RB{}\+\\*{}%
    e \assign{} r\-\\*{}%
  \kw{end} getReal\\*{}%
  \kw{export} \kw{function} getImag \returns{} \/\LB{}e \CO{} \tn{Real}\/\RB{}\+\\*{}%
    e \assign{} i\-\\*{}%
  \kw{end} getImag\-\\*{}%
\kw{end} Complex
\end{tabbing}\end{minipage}}

This is rearranged into the following:

{\small\it\begin{minipage}{\textwidth}\begin{tabbing}
xxx\=xxx\=xxx\=xxx\=xxx\=xxx\=xxx\=xxx\=xxx\=xxx\=xxx\=xxx\=xxx\=\+\kill%
\kw{const} Complex \assign{} \kw{immutable} \kw{object} Complex\+\\*{}%
  \kw{const} ComplexType \assign{} \kw{immutable} \kw{typeobject} ComplexType\+\\*{}%
    \kw{function} $+$\/\LB{}ComplexType\/\RB{} \returns{} \/\LB{}ComplexType\/\RB{}\\*{}%
    \kw{function} getReal \returns{} \/\LB{}ComplexType\/\RB{}\\*{}%
    \kw{function} getImag \returns{} \/\LB{}ComplexType\/\RB{}\-\\*{}%
  \kw{end} ComplexType\\*{}%
  \kw{export} \kw{function} getSignature \returns{} \/\LB{}r \CO{} \tn{Signature}\/\RB{}\+\\*{}%
    r \assign{} ComplexType\-\\*{}%
  \kw{end} getSignature\\*{}%
  \kw{export} \kw{operation} fromReal\/\LB{}a \CO{} \tn{Real}\/\RB{} \returns{} \/\LB{}e \CO{} Complex\/\RB{}\+\\*{}%
    e \assign{} self.create\/\LB{}a, 0.0\/\RB{}\-\\*{}%
  \kw{end} fromReal\\*{}%
  \kw{export} \kw{operation} create\/\LB{}r \CO{} \tn{Real}, i \CO{} \tn{Real}\/\RB{} \returns{} \/\LB{}e \CO{} Complex\/\RB{}\+\\*{}%
    e \assign{} \kw{immutable} \kw{object} aComplex\+\\*{}%
      \kw{export} \kw{function} $+$\/\LB{}other \CO{} Complex\/\RB{} \returns{} \/\LB{}e \CO{} Complex\/\RB{}\+\\*{}%
	e \assign{} Complex.create\/\LB{}other.getReal $+$ r, other.getImag $+$ i\/\RB{}\-\\*{}%
      \kw{end} $+$\\*{}%
      \kw{export} \kw{function} getReal \returns{} \/\LB{}e \CO{} \tn{Real}\/\RB{}\+\\*{}%
	e \assign{} r\-\\*{}%
      \kw{end} getReal\\*{}%
      \kw{export} \kw{function} getImag \returns{} \/\LB{}e \CO{} \tn{Real}\/\RB{}\+\\*{}%
	e \assign{} i\-\\*{}%
      \kw{end} getImag\-\\*{}%
    \kw{end} aComplex\-\\*{}%
  \kw{end} create\-\\*{}%
\kw{end} Complex
\end{tabbing}\end{minipage}}

\subsubsection*{Inheritance}
Emerald supports {\em single inheritance}.  That is, every class may have at
most one superclass from which it inherits.  There are three kinds of
components that may be
inherited from the parent class:
\begin{itemize}
  \item{} class operations
  \item{} instance declarations (constants and variables)
  \item{} instance operations including any initially, recovery, and process
  \item{} parameters to the class.
\end{itemize}
The parameters to the subclass are concatenated to the end of the list of
parameters to the
superclass in order to form the final parameter list for the subclass.

Inheritance of the other three kinds of components is performed by
considering in turn each component of the 
superclass, and searching for an identically named component of the same
kind in the subclass.  If the subclass contains the component then the
superclass component is ignored, otherwise the superclass component is added
to the subclass.

There is no support for changing the visibility of a component (exporting in
the subclass an
operation that is private in the superclass or making private in the
subclass an operation
that is exported in the superclass), nor is there support for deleting a
component.  Because subclass components completely replace those from the
superclass, changing the visibility of a component may be accomplished by
copying the component from the superclass manually and changing the
visibility of the copy.
A data component may be effectively deleted by declaring an
identically named constant in the subclass whose value is sufficiently
simple that it need not be stored (an excellent choice of value is 0).

Because any component of the superclass may be redefined in the subclass,
there is no guarantee that either the type of the subclass will conform to
the type of the superclass, or that the type of instances of the subclass
will conform to the type of instances of the superclass.
The objects resulting from inheritance do not
retain at run time any record of their inheritance relationships.  That is,
there is no operation that can be performed on a class object to retrieve
its superclass.  This is a
side effect of the fact that there is no type {\it class}.  Such operations
can easily be implemented, if desired, as in the following example:

{\small\it\begin{minipage}{\textwidth}\begin{tabbing}
xxx\=xxx\=xxx\=xxx\=xxx\=xxx\=xxx\=xxx\=xxx\=xxx\=xxx\=xxx\=xxx\=\+\kill%
\kw{const} parent \assign{} \kw{class} parent \\*{}%
\kw{end} parent\\[1.0ex]{}%
\kw{const} child \assign{} \kw{class} child (parent)\+\\*{}%
  \kw{class} \kw{export} \kw{function} getSuperClass \returns{} \/\LB{}r \CO{} \tn{Any}\/\RB{}\+\\*{}%
    r \assign{} parent\-\\*{}%
  \kw{end} getSuperClass\-\\*{}%
\kw{end} child
\end{tabbing}\end{minipage}}
  

\subsection{Enumerations}
\begin{quote}\it\begin{tabular}{lcl}
enum &$::=$& \kw{enumeration} identifier\\
& & \hspace{0.5in} \sseq{enumIdentifier}{\terminal{,}} \\
& & \kw{end} identifier
\end{tabular}\end{quote}
An enumeration is a class that represents an ordered collection of
identifiers.  The operations on the class consist of:
\begin{itemize}
  \item{} a function getSignature, which returns a \tn{Signature} describing
    the type of the enumeration instances
  \item{} for each enumeration identifier $a$, a creation operation named
    $a$ that returns an object representing that element of the enumeration
  \item{} {\it first\/} and {\it last\/} that return the first and last
    elements of the enumeration, respectively
  \item{} an operation named {\it create} that takes an integer argument
    {\it n} and returns the {\it n}\/th element of the enumeration, with the
    numbering starting at 0.
\end{itemize}
Each instance of the class implements the following operations:
\begin{itemize}
  \item{} the comparison functions $<$, $<=$, $=$, $!=$, $>=$, $>$
  \item{} functions {\it succ\/} which returns the successor object (or
    fails if invoked on the last element of the enumeration) and {\it
    pred\/} which returns the predecessor object (or fails if invoked on the
    first element of the enumeration)
  \item{} a function {\it ord\/} which returns the position of the element
    in the enumeration ordering, starting at 0
  \item{} a function {\it asString} which returns the name of the element as
    a \tn{String}.
\end{itemize}
    
All instance of enumeration classes are immutable.
To be concrete, consider the declaration:

{\small\it\begin{minipage}{\textwidth}\begin{tabbing}
xxx\=xxx\=xxx\=xxx\=xxx\=xxx\=xxx\=xxx\=xxx\=xxx\=xxx\=xxx\=xxx\=\+\kill%
\kw{const} colors \assign{} \kw{enumeration} colors red, blue, green \kw{end} colors
\end{tabbing}\end{minipage}}

The class object {\it colors\/} will have type:

{\small\it\begin{minipage}{\textwidth}\begin{tabbing}
xxx\=xxx\=xxx\=xxx\=xxx\=xxx\=xxx\=xxx\=xxx\=xxx\=xxx\=xxx\=xxx\=\+\kill%
\kw{immutable} \kw{typeobject} ColorCreatorType\+\\*{}%
  \kw{function}  getSignature \returns{} \/\LB{}\tn{Signature}\/\RB{}\\*{}%
  \kw{operation} create\/\LB{}\tn{Integer}\/\RB{} \returns{} \/\LB{}ColorType\/\RB{}\\*{}%
  \kw{operation} first \returns{} \/\LB{}ColorType\/\RB{}\\*{}%
  \kw{operation} last \returns{} \/\LB{}ColorType\/\RB{}\\*{}%
  \kw{operation} red \returns{} \/\LB{}ColorType\/\RB{}\\*{}%
  \kw{operation} green \returns{} \/\LB{}ColorType\/\RB{}\\*{}%
  \kw{operation} blue \returns{} \/\LB{}ColorType\/\RB{}\-\\*{}%
\kw{end} ColorCreatorType
\end{tabbing}\end{minipage}}

and each element of the enumeration will have type:

{\small\it\begin{minipage}{\textwidth}\begin{tabbing}
xxx\=xxx\=xxx\=xxx\=xxx\=xxx\=xxx\=xxx\=xxx\=xxx\=xxx\=xxx\=xxx\=\+\kill%
\kw{immutable} \kw{typeobject} ColorType\+\\*{}%
  \kw{function} $<$\/\LB{}ColorType\/\RB{} \returns{} \/\LB{}\tn{Boolean}\/\RB{}\\*{}%
  \kw{function} $<=$\/\LB{}ColorType\/\RB{} \returns{} \/\LB{}\tn{Boolean}\/\RB{}\\*{}%
  \kw{function} $=$\/\LB{}ColorType\/\RB{} \returns{} \/\LB{}\tn{Boolean}\/\RB{}\\*{}%
  \kw{function} $!=$\/\LB{}ColorType\/\RB{} \returns{} \/\LB{}\tn{Boolean}\/\RB{}\\*{}%
  \kw{function} $>=$\/\LB{}ColorType\/\RB{} \returns{} \/\LB{}\tn{Boolean}\/\RB{}\\*{}%
  \kw{function} $>$\/\LB{}ColorType\/\RB{} \returns{} \/\LB{}\tn{Boolean}\/\RB{}\\*{}%
  \kw{function} succ \returns{} \/\LB{}ColorType\/\RB{}\\*{}%
  \kw{function} pred \returns{} \/\LB{}ColorType\/\RB{}\\*{}%
  \kw{function} ord \returns{} \/\LB{}\tn{Integer}\/\RB{}\\*{}%
  \kw{function} asString \returns{} \/\LB{}\tn{String}\/\RB{}\-\\*{}%
\kw{end} ColorType
\end{tabbing}\end{minipage}}

\subsection{Fields}
\begin{quote}\it\begin{tabular}{lcl}
field &$::=$& 
  \opt{\kw{attached}} \kw{const} \kw{field} identifier \terminal{:} type initializer\\
& $|$ &  \opt{\kw{attached}} \kw{field} identifier \terminal{:} type \opt{initializer}
\end{tabular}\end{quote}

It is often convenient to declare an externally accessible data element of
\label{fields}
an object.  A field declaration does exactly this.  Field declarations can
only occur within the declaration part of an object constructor.
Constant field declarations expand to a constant declaration and an operation
to get the value of the constant.  Variable fields expand to a variable
declaration and operations to both get and set the value of the variable.
The expansion of the constant field: 

{\small\it\begin{minipage}{\textwidth}\begin{tabbing}
xxx\=xxx\=xxx\=xxx\=xxx\=xxx\=xxx\=xxx\=xxx\=xxx\=xxx\=xxx\=xxx\=\+\kill%
\kw{attached} \kw{const} \kw{field} f \CO{} t \assign{} init
\end{tabbing}\end{minipage}}

is

{\small\it\begin{minipage}{\textwidth}\begin{tabbing}
xxx\=xxx\=xxx\=xxx\=xxx\=xxx\=xxx\=xxx\=xxx\=xxx\=xxx\=xxx\=xxx\=\+\kill%
\kw{attached} \kw{const} f \CO{} t \assign{} init\\*{}%
\kw{export} \kw{function} getF \returns{} \/\LB{}x \CO{} t\/\RB{}\+\\*{}%
  x \assign{} f\-\\*{}%
\kw{end} getF
\end{tabbing}\end{minipage}}

And the expansion of the variable field:

{\small\it\begin{minipage}{\textwidth}\begin{tabbing}
xxx\=xxx\=xxx\=xxx\=xxx\=xxx\=xxx\=xxx\=xxx\=xxx\=xxx\=xxx\=xxx\=\+\kill%
\kw{attached} \kw{field} f \CO{} t \assign{} init
\end{tabbing}\end{minipage}}

is

{\small\it\begin{minipage}{\textwidth}\begin{tabbing}
xxx\=xxx\=xxx\=xxx\=xxx\=xxx\=xxx\=xxx\=xxx\=xxx\=xxx\=xxx\=xxx\=\+\kill%
\kw{attached} \kw{var} f \CO{} t \assign{} init\\*{}%
\kw{export} \kw{operation} setF\/\LB{}x \CO{} t\/\RB{}\+\\*{}%
  f \assign{} x\-\\*{}%
\kw{end} setF\\*{}%
\kw{export} \kw{function} getF \returns{} \/\LB{}x \CO{} t\/\RB{}\+\\*{}%
  x \assign{} f\-\\*{}%
\kw{end} getF
\end{tabbing}\end{minipage}}

\subsection{Records}

\begin{quote}\it\begin{tabular}{lcl}
record &$::=$& \opt{\kw{immutable}} \kw{record} identifier \\
& & \hspace{0.5in} \seq{recordfield} \\
& & \kw{end} identifier \\
recordfield & $::=$ & \opt{\kw{attached}} \opt{\kw{var}} fieldIdentifier : type
\end{tabular}\end{quote}

A record is a class that contains a field for each element of the record,
and a {\it create} function that takes as its parameters initial values for
each field.  If the record is defined as immutable, then the class will be
immutable as well, and each field will be a constant field.  For example,
the declaration of the mutable record:

{\small\it\begin{minipage}{\textwidth}\begin{tabbing}
xxx\=xxx\=xxx\=xxx\=xxx\=xxx\=xxx\=xxx\=xxx\=xxx\=xxx\=xxx\=xxx\=\+\kill%
\kw{record} aRecord\+\\*{}%
  a \CO{} Integer\\*{}%
  c \CO{} String\-\\*{}%
\kw{end} aRecord
\end{tabbing}\end{minipage}}

expands to a class:

{\small\it\begin{minipage}{\textwidth}\begin{tabbing}
xxx\=xxx\=xxx\=xxx\=xxx\=xxx\=xxx\=xxx\=xxx\=xxx\=xxx\=xxx\=xxx\=\+\kill%
\kw{class} aRecord\/\LB{}xa \CO{} Integer, xc \CO{} String\/\RB{}\+\\*{}%
  field a \CO{} Integer \assign{} xa\\*{}%
  field c \CO{} String \assign{} xc\-\\*{}%
\kw{end} aRecord
\end{tabbing}\end{minipage}}

Given the above declaration, the following code
declares and initializes a record variable.

{\small\it\begin{minipage}{\textwidth}\begin{tabbing}
xxx\=xxx\=xxx\=xxx\=xxx\=xxx\=xxx\=xxx\=xxx\=xxx\=xxx\=xxx\=xxx\=\+\kill%
  \kw{var} a \CO{} aRecord\\[1.0ex]{}%
  a \assign{} aRecord.create\/\LB{}34, ``A string''\/\RB{}
\end{tabbing}\end{minipage}}

\subsection{Predefined objects}
\emd{} implements a number of pre-defined
objects; these objects are outlined in Table~\ref{BuiltinsTable}
and specified in greater detail in
Appendix~\ref{builtin objects}.


\begin{table} 
\begin{center}
\it
\begin{tabular}{||l|p{3.5in}||}
    \hline
Predefined Type           & Type Description \\
    \hline
    \hline
\tn{Any}		& Has no operations.\\
\tn{Array}		& A polymorphic, flexible array.\\
\tn{AOpVector}		& A sequence of operation signatures in a Signature.\\
\tn{AOpVectorE}		& A single operation signature in a Signature.\\
\tn{AParamList}		& A list of parameters in an operation Signature.\\
\tn{BitChunk}		& A container of bits supporting bit-level operations.\\
\tn{Boolean}		& Logical values with literals \kw{true} 
			  and \kw{false}.\\
\tn{Character}		& Individual characters with operations such as
			  $<$, $>$, $=$, ord, etc.\\
\tn{ConcreteType}	& A container for the executable code of an object.\\
\tn{Condition} 		& Condition variables satisfying Hoare monitor semantics.\\
\tn{COpVector}		& A sequence of operation definitions in a ConcreteType.\\
\tn{COpVectorE}		& A single operation definition in a ConcreteType.\\
\tn{Directory}		& An object defining the type of primitive name server directories.\\
\tn{Handler}		& An object defining the type of objects capable of receiving Node state change updates from the run time system.\\
\tn{InterpreterState}	& An internal object capturing the state of the execution of a process.\\
\tn{ImmutableVector}    & Read-only vector.\\
\tn{ImmutableVectorOfAny}    & Read-only vector of Any.\\
\tn{ImmutableVectorOfInt}    & Read-only vector of Integers.\\
\tn{InStream}		& Input streams.\\
\tn{Integer} 		& Signed integers.\\
\tn{Node}		& Objects representing machines.\\
\tn{NodeList}           & Immutable vectors of node descriptions.\\
\tn{NodeListElement}	& Immutable node descriptions.\\
\tn{None}		& The type of \kw{nil}.\\
\tn{OutStream}		& Output streams.\\
\tn{Real}		& Approximations of real numbers.\\
\tn{RISA}		& Readable Indexed Sequence of Any.\\
\tn{RISC}		& Readable Indexed Sequence of Character.\\
\tn{Signature}		& Primitive abstract type\\
\tn{String}		& Character strings.\\
\tn{Time}		& Times and dates\\
\tn{Type}		& The type of all types.\\
\tn{Vector}		& Fixed sized polymorphic vectors.\\
\tn{VectorOfChar}	& Vector.of[Character].\\
\tn{VectorOfInt}	& Vector.of[Integer].\\
    \hline
\end{tabular}
\end{center}
\caption{Built-in Types}
\label{BuiltinsTable}
\end{table}
