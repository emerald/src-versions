\section{Statements}
\label{statements}
\subsection{Assignment statement}
\label{assignment statement}
\begin{quote}\it\begin{tabular}{lcl}
assignment  & $::=$   & identifierList \terminal{\assign} expressionList \\
 & $|$ & \opt{identifierList \terminal{\assign}} procedureInvocation
\end{tabular}\end{quote}
In the first case, the expression list is evaluated to yield a number of
objects. In the latter case, the procedure invocation is performed, resulting
in a number of objects (possibly 0). In both cases, the resulting objects are
positionally bound to the variables on the
left side of the assignment operator. The number of variables on the left
side and the number of resulting objects on the right must be equal and must
positionally conform in type (see Section~\ref{conformity}).

\subsection{Selection}
\label{selection stmt}
\begin{quote}\it\begin{tabular}{lcl}
ifStatement & $::=$ & \kw{if} expression \kw{then} \\
            &    & \hspace{0.5in}declarationsAndStatements \\
            &    & \{ \kw{elseif} expression \kw{then} \\
            &    & \hspace{0.5in} declarationsAndStatements \} \\
            &    & {\rm [} \kw{else}\\
            &    & \hspace{0.5in} declarationsAndStatements {\rm ]} \\
            &    & \kw{end} \kw{if}
\end{tabular}\end{quote}
The expressions following the \kw{if} and optional \kw{elseif} 
keywords (which must be of type \tn{Boolean}) are evaluated in textual
order until one evaluates to \kw{true} or they are exhausted. In the
former case, the statements following the next \kw{then} keyword are
executed, and in the latter case, the statements following
the \kw{else} keyword (when present) are executed.

\subsection{Iteration}
\subsubsection{Loop statement}
\label{loop stmt}
\begin{quote}\it\begin{tabular}{lcl}
loopStatement   & $::=$ & \kw{loop} \\
                &     & \hspace{0.5in} declarationsAndStatements \\
                &     & \kw{end} \kw{loop}
\end{tabular}\end{quote}
The statements bracketed by \kw{loop} and \kw{end} \kw{loop} are executed
repeatedly until an exit statement at the same level of nesting is executed.

\subsubsection{Exit statement}
\begin{quote}\it\begin{tabular}{lcl}
exitStatement & $::=$ &   \kw{exit} \opt{\kw{when} expression}
\end{tabular}\end{quote}
This statement terminates the execution of the  textually inner-most
enclosing loop; this statement is invalid if there is no such loop. The
simple \kw{exit} provides an unconditional exit from the loop; the optional
\kw{when} clause
permits a conditional exit if the evaluated expression, which must be of type
\tn{Boolean}, evaluates to \kw{true}.

\subsubsection{For statement}
\label{for stmt}
Emerald has two forms of the for statement.  These are conveniences whose
semantics are defined in terms of their translations as given below.

\begin{quote}\it\begin{tabular}{lcl}
forStatement   & $::=$ & \kw{for} \terminal{(} initial \terminal{:} condition \terminal{:}  step \terminal{)} \\
                &     & \hspace{0.5in} declarationsAndStatements \\
                &     & \kw{end} \kw{for}
\end{tabular}\end{quote}
\noindent
This is equivalent to:
\begin{quote}\it
{\it\begin{minipage}{\textwidth}\begin{tabbing}
xxx\=xxx\=xxx\=xxx\=xxx\=xxx\=xxx\=xxx\=xxx\=xxx\=xxx\=xxx\=xxx\=\+\kill%
  \kw{begin}\+\\*{}%
    initial\\*{}%
    \kw{loop}\+\\*{}%
      \kw{exit} \kw{when} $!$condition\\*{}%
      \kw{begin}\+\\*{}%
	declarationsAndStatements\-\\*{}%
      \kw{end}\\*{}%
      step\-\\*{}%
    \kw{end} \kw{loop}\-\\*{}%
  \kw{end}
\end{tabbing}\end{minipage}}
\end{quote}

\begin{quote}\it\begin{tabular}{lcl}
forStatement & $::=$ & \kw{for} identifier \terminal{:} typeExpression initialization \kw{while} condition \kw{by} step\\
                &     & \hspace{0.5in} declarationsAndStatements \\
                &     & \kw{end} \kw{for}
\end{tabular}\end{quote}
\noindent
This is equivalent to
\begin{quote}\it
{\it\begin{minipage}{\textwidth}\begin{tabbing}
xxx\=xxx\=xxx\=xxx\=xxx\=xxx\=xxx\=xxx\=xxx\=xxx\=xxx\=xxx\=xxx\=\+\kill%
  \kw{begin}\+\\*{}%
    \kw{var} identifier : typeExpression initialization \\*{}%
    \kw{loop}\+\\*{}%
      \kw{exit} \kw{when} $!$condition\\*{}%
      \kw{begin}\+\\*{}%
	declarationsAndStatements\-\\*{}%
      \kw{end}\\*{}%
      step\-\\*{}%
    \kw{end} \kw{loop}\-\\*{}%
  \kw{end}
\end{tabbing}\end{minipage}}
\end{quote}

\subsection{Compound statement}
\label{compound statement}
\begin{quote}\it\begin{tabular}{lcl}
compoundStatement &$::=$&\kw{begin} \\
		  &   &\hspace{0.5in}blockBody \\
                  &   &\kw{end}\\[1ex]
blockBody         &$::=$&declarationsAndStatements \\
	          &   & \opt{unavailableHandler} \\
                  &   & \opt{failureHandler}
\end{tabular}\end{quote}
The compound statement permits several statements to be grouped together as
one composite statement. In addition, it permits suitable recovery code to
be attached in the form of handlers dealing with object unavailability
and failures (cf.~Sections~\ref{unavailable objects} and \ref{failures}).

\subsection{Assertions}
\begin{quote}\it\begin{tabular}{lcl}
assertStatement & $::=$ & \kw{assert} expression
\end{tabular}\end{quote}
The expression, whose type must be \tn{Boolean}, is evaluated.
If the result is \kw{false}, a failure occurs (as explained in
Section~\ref{failures}).  If the result is \kw{true}, the statement has no
further effect.

\subsection{Concurrency}
Concurrency features are described in detail in Section~\ref{objects} and
are briefly outlined here.
Each object may have an optional process associated with it; this
process is created after the termination of the object's \kw{initially} section
and it executes until it reaches the end of its block.
Any object which is defined as \kw{monitor} guarantees mutual exclusion in
the execution of all of its operations.  Objects of system-implemented type
\tn{Condition} 
may be used for synchronization within monitored objects; the semantics of
condition waits and signals follows that proposed by Hoare.  Note that a
condition object used in a wait or signal statement or an
awaiting expression must be used only inside the monitored object
by which it was created.

\subsubsection{Wait statement}
\begin{quote}\it\begin{tabular}{lcl}
waitStatement &$::=$& \kw{wait} expression
\end{tabular}\end{quote}
The wait statement must be executed inside a monitored object, and
the type of the {\em expression} must be \tn{Condition}.
The process executing the wait is suspended on the condition
object, and the monitor
lock is passed on to the next process waiting to enter the monitor;
if no process is waiting to enter, the monitor lock is released.

\subsubsection{Signal statement}
\begin{quote}\it\begin{tabular}{lcl}
signalStatement &$::=$& \kw{signal} expression
\end{tabular}\end{quote}
The type of the {\it expression} must be \tn{Condition}.
If the condition object has one or more processes suspended on it,
one of these processes will be resumed, the monitor lock will be
passed to it, and the signalling process will be placed at the head of the
monitor entry queue.
If the condition object does not have any processes suspended on it,
the signal statement has no effect.

\subsection{Location-related Statements}
Mobility is an important feature of \emd{} (\cite{Jul88thesis,Jul88tocs})
and is supported via several language constructs. The
statements that permit the programmer to specify and change the location of
objects are discussed below.
\subsubsection{Fix statement}
\label{fix statement}
\begin{quote}\it\begin{tabular}{lcl}
fixStatement &$::=$& \kw{fix} expression$_{1}$ \kw{at} expression$_{2}$
\end{tabular}\end{quote}
The object named by {\it expression}$_{1}$ is moved to the location of
the object named by {\it expression}$_{2}$, and forced to remain there; the
{\em unfix} and {\em refix} statements described below permit the movement
of previously fixed objects.
Attempts to move or fix
previously fixed objects result in failures (cf.~Section~\ref{failures}).

\subsubsection{Unfix statement}
\label{unfix statement}
\begin{quote}\it\begin{tabular}{lcl}
unfixStatement &$::=$& \kw{unfix} expression
\end{tabular}\end{quote}
The object denoted by the {\it expression} is made free to move.  It is not an
error to unfix an object not currently fixed at any location.

\subsubsection{Refix statement}
\label{refix statement}
\begin{quote}\it\begin{tabular}{lcl}
refixStatement &$::=$& \kw{refix} expression$_{1}$ \kw{at} expression$_{2}$
\end{tabular}\end{quote}
This statement unfixes the object named by $expression_{1}$ and fixes it
at some (presumably different) location; the refix is performed atomically.

\subsubsection{Move statement}
\label{move statement}
\begin{quote}\it\begin{tabular}{lcl}
moveStatement &$::=$& \kw{move} expression$_{1}$ \kw{to} expression$_{2}$
\end{tabular}\end{quote}
The object denoted by {\it expression}$_{1}$ is moved
to the current location of the
object denoted by {\it expression}$_{2}$.  The statement fails
if the object denoted by {\it expression}$_{1}$ is fixed.
The \kw{move} primitive is actually a hint, i.e., the implementation
is not required to perform the move suggested. On the other hand,
the primitives \kw{fix} and \kw{refix} have stronger semantics, and when they
succeed, the object must stay at the specified destination until explicitly
unfixed or refixed.

\subsection{The Checkpoint statement}
\begin{quote}\it\begin{tabular}{lcl}
checkpointStatement &$::=$& \kw{checkpoint}
\end{tabular}\end{quote}
The checkpoint statement permits an object to store its state on permanent
storage. On node failure and subsequent recovery, the object uses this
stored state and continues from that state, first performing any
programmer-specified
recovery action.

\subsection{The Return Statement}
\begin{quote}\it\begin{tabular}{lcl}
returnStatement &$::=$& \kw{return}
\end{tabular}\end{quote}
This statement is used to terminate the execution of an operation and return
to the invoking object.  It may also be used to prematurely terminate an
initially, process, or recovery section.

\subsection{The ReturnAndFail statement}
\begin{quote}\it\begin{tabular}{lcl}
returnAndFailStatement &$::=$& \kw{returnandfail}
\end{tabular}\end{quote}
The return and fail statement is analogous to the return statement, but
in addition, it permits the invoked object to report a failure to the invoking
object.  The return happens first so the state of the invoked object is not
affected by the failure (cf.~Section~\ref{failures}).

\subsection{The Primitive Statement}
\begin{quote}\it\begin{tabular}{lcl}
primitiveStatement &$::=$& \kw{primitive} \opt{\kw{self}} \opt{\kw{var}}
primitiveImplementation \\
&&\terminal{[} identifierList \terminal{]} \assign \terminal{[}
identifierList \terminal{]} \\
primitiveImplementation &$::=$& \oseq{stringLiteral $|$ integerLiteral}
\end{tabular}\end{quote}
This statement is used to implement lower-level calls to the underlying
operating system and to implement certain operations on builtin-types.
Primitive statements are used only in the implementation of the
builtin Emerald types and should not be used by the Emerald programmer
directly.  

The optional \kw{self} means that the receiving object ought to be
pushed onto the stack before the argument variables, otherwise it is not
pushed.  The optional \kw{var} means that all the things pushed onto the stack
and all the results are assumed to be variables represented as 2 words:  a
data pointer and a concrete type pointer.  Otherwise all arguments and
results are assumed to just be data (1 word).  The strings and integers
in the primitiveImplementation are placed in
the instruction stream as a sequence of bytes.  Strings are looked up as
either names of bytecodes in ``..../lib/bcdef'' or system defined operations
in ``..../lib/jsdef'' or ``..../lib/ccdef''.
The list of identifiers on the right provide the
arguments for the primitive, while the list of identifiers on the left
get the results of the primitive.  The following code is
generated:

\begin{itemize}
  \item{} If self is present, the push the receiving object onto the stack
  \item{} Push the value of each of the argument identifiers onto the stack,
      left to right 
  \item{} Emit the literals in the primitive implementation into the
      instruction stream
  \item{} Pop results off the stack storing them in the result identifiers,
      left to right 
\end{itemize}

There is no check that the implementation of the primitive actually expects
the number of arguments or returns the number of results that the
argument and result identifier lists mention.
